\documentclass[pdftex,12pt,letter]{article}
\usepackage[margin=0.75in]{geometry}
\usepackage{verbatim}
\usepackage{graphicx}
\usepackage{xspace}
\usepackage{cite}
\usepackage{url}
\usepackage[pdftex,pdfpagelabels,bookmarks,hyperindex,hyperfigures]{hyperref}


\title{Modern architectures of the Conditions Database Systems and their potential applications in DUNE}
\date{\today}
\author{P.\,Laycock, M.\,Potekhin}


\begin{document}
\maketitle

\begin{abstract}
\noindent  In the past few years the HEP community has developed fresh understanding of
the use cases and challenges of the Conditions Database systems based on the experience of
the LHC and other experiments. New and converging designs have been created and implemented
in experiments such as CMS, Belle II and others. We consider the features of these
new architectures and potential advantages they can bring to DUNE.

%% \cite{docdb1086}, DocDB\ and
\end{abstract}

%%%%%%%%%%%%%%%%%%%%%%%%%%%%%%%%%%%%%%%%%%%%%%%%%%%%%

\section{Overview}
According to  the HSF document titled \textit{``A Roadmap for HEP Software and Computing R\&D for the 2020s''} \cite{hsf_roadmap},
\begin{quote}
Conditions data is defined as the non-event data required by data-processing software to correctly simulate,
digitise or reconstruct the raw detector event data. The non-event data discussed here consists mainly of
detector calibration and alignment information, with some additional data describing the detector configuration,
the machine parameters, as well as information from the detector control system.
\end{quote}
\noindent It is important to note that the conditions data comprises a considerable variety of information,
which may also come in different formats and in different volumes depending on the use case. In the following
we will use the abbreviated term \textbf{CDB } to refer to the conditions database both as a concept and as an
instance where appropriate.

An important feature of the operating environment of a CDB is a potentially very high
access rate due to a large number of processing jobs runnning simultaneously in the distrubuted environment
and requesting the conditions data. Use of some sort of efficient caching mechanism is mandatory for
the system to be performant.


\subsection{Architecture Convergence}
HEP and NP experiments have been using a variety of CDB systems over the past few decades.
More recently, the LHC experiments encountered issues related to the complexity and
maintainability \cite{func_test}. To meet the challenges of the ever increasing scale of
data processing the LHC experiments such as ATLAS and CMS have converged on
a common CDB architecture, which is also now shared by Bells II.

In this approach, there is a separation of the content (commonly termed \textit{payload})


%% \begin{figure}[tbh]
%%   \centering
%%   \includegraphics[width=1.1\textwidth]{figures/prompt_queues_1.pdf}
%%   \caption{Categories of prompt processing jobs grouped in queues. In this example, the depths of queues are different for different job types.}
%%   \label{fig:queues1}
%% \end{figure}


%%\newpage
%%\appendix
%%\section{Glossary}
%%\label{sec:glossary}
%%\begin{description}
%%\item [foo] description
%%\end{description}


\clearpage
\begin{thebibliography}{1}

\bibitem{hsf_roadmap}
{\textit{A Roadmap for HEP Software and Computing R\&D for the 2020s}arXiv:1712.06982} \url{https://arxiv.org/abs/1712.06982}

\bibitem{func_test}
{\textit{Functional tests of a prototype for the CMS-ATLAS common non-event data handling framework}} \url{https://iopscience.iop.org/article/10.1088/1742-6596/898/4/042047}

\bibitem{condb}
{\textit{Conditions Database at Fermilab}  \url{https://cdcvs.fnal.gov/redmine/projects/condb/wiki/Conditions_Database_at_Fermilab}}

\end{thebibliography}


\end{document}
