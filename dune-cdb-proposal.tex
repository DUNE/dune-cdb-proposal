\documentclass[pdftex,12pt,letter]{article}
\usepackage[margin=0.75in]{geometry}
\usepackage{verbatim}
\usepackage{graphicx}
\usepackage{xspace}
\usepackage{cite}
\usepackage{url}
\usepackage[pdftex,pdfpagelabels,bookmarks,hyperindex,hyperfigures]{hyperref}


\title{A Design of the \pd/\expname Prompt Processing System}
\date{\today}
\author{N. Benekos, M.\,Potekhin and B.\,Viren}


\begin{document}
\maketitle

\begin{abstract}
\noindent  We start with a brief summary

%% \cite{docdb1086}, DocDB\ and
\end{abstract}

%%%%%%%%%%%%%%%%%%%%%%%%%%%%%%%%%%%%%%%%%%%%%%%%%%%%%

\section{Overview}
\subsection{Terminology}
Some of the terms, acronyms and abbreviations used in this document are explained in the glossary in Appendix %% \ref{sec:glossary}.

\subsection{Motivation}
\underlineDUNE

\subsection{Data Scenarios and Data Handling}
\label{sec:rawdata}

\subsection{Data Streams}
\label{sec:streams}

For reasons that are outside of the scope of this document  the experimental raw data
will come as three distinct streams at the time they are captured:
\begin{itemize}
\item \textbf{LArTPC} The TPC  stream which is dominant in terms of rate and volume
\item \textbf{BI} Beam Instrumentation data from Cherenkov, TOF and other systems
\item \textbf{CRT} Cosmic Ray Tagger data
\end{itemize}

\noindent All three streams will follow a similar path in that they will be transmitted from their respective
online buffers.


%% \begin{figure}[tbh]
%%   \centering
%%   \includegraphics[width=1.1\textwidth]{figures/prompt_queues_1.pdf}
%%   \caption{Categories of prompt processing jobs grouped in queues. In this example, the depths of queues are different for different job types.}
%%   \label{fig:queues1}
%% \end{figure}


\newpage
\appendix
\section{Glossary}
\label{sec:glossary}
\begin{description}

\item [foo] description

\end{description}


\clearpage
\begin{thebibliography}{1}
\bibitem{ref1}
{ref1: \textit{Reference 1 }\\
\url{http://ref1.com}

\end{thebibliography}


\end{document}
